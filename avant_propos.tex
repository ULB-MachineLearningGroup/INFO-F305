\chapter*{Avant-propos}
    Le présent document vise à offrir un support à l'étudiant·e dans la réalisation des exercices en séances de travaux pratiques pour le cours de Modélisation et Simulation donné en troisième année de bachelier à l'Université Libre de Bruxelles. Il sert de complément au syllabus du cours magistral, véritable outil de référence pour l'apprentissage.
    Le contenu de ce syllabus se base sur les ressources préexistantes, remises en forme et complétées d'explications et de mises en contextes historiques et scientifiques. L'objectif de ce texte est à la fois de donner à l'étudiant·e toutes les clés nécessaires à la réussite de la partie pratique du cours, mais surtout, de montrer l'étendue des possibilités offertes par le monde de la simulation numérique.

    Ce document n'est pas extensif: bon nombre de points n'y sont pas abordés, plusieurs autres sont supposés connus. L'étudiant·e est invité·e, avant tout, à se référer au syllabus principal, et à la documentation des outils présentés dans les différents chapitres.

    Les outils présentés étant en constante évolution, il n'est pas impossible que certains points présentés dans les prochains chapitres deviennent obsolètes; les auteurs, néanmoins, sont convaincus que la plupart des concepts entourant l'utilisation des outils spécifiques resteront utiles.

    Enfin, l'ouvrage est aussi à utiliser comme un \textit{mémo pratique}: les exercices résolus, les exemples illustrés, accompagnés des codes pour les générer, seront sûrement d'utilité non négligeable pour rapidement se remémorer les méthodes de résolution et de simulation.

    Nous voulons remercier, sans ordre d'importance, les personnes suivantes, pour la rédaction de ce document et pour les ressources mises à disposition: \textit{Gianluca Bontempi, Abel Laval, Jacopo De Stefani, Robin Petit, Yannick Molinghen}.