\chapter{Conclusion}
    Ce syllabus a été conçu pour vous initier à l'art de la Modélisation et de la Simulation numérique des systèmes dynamiques. À travers les chapitres abordés, vous avez appris à résoudre différents types d'équations différentielles, à analyser des comportements de systèmes complexes, et à simuler numériquement des processus dynamiques en utilisant des outils informatiques.
    
    En modélisant divers phénomènes et en apprenant à interpréter les résultats des simulations, vous avez découvert comment les mathématiques et l'informatique peuvent nous permettre de représenter et de manipuler des systèmes complexes. Ces compétences vous placent maintenant en excellente position pour aborder des domaines encore plus avancés, où l'incertitude et les données jouent un rôle central.
    
    La suite naturelle de ce cours se trouve au sein du cours de \textit{Statistical Foundations of Machine Learning}. Dans ce cours, vous passerez de la simulation numérique et de la modélisation déterministe à l'inférence statistique et à la recherche de modèles prédictifs. Vous y apprendrez à modéliser des comportements non plus uniquement à partir de lois physiques, mais à partir de données, en employant des techniques statistiques pour estimer les paramètres et structures des modèles.
    
    Ce passage de la modélisation de systèmes déterministes à l'inférence statistique vous permettra de développer une compréhension encore plus fine des processus complexes, en apprenant à exploiter les données pour construire des modèles prédictifs et explicatifs. Ainsi, vous serez équipé pour répondre aux défis modernes posés par les grands ensembles de données et les systèmes incertains.
    
    Nous espérons que ce syllabus vous a non seulement transmis des connaissances pratiques et théoriques, mais aussi une intuition pour explorer plus avant le monde des systèmes dynamiques. Continuez à expérimenter, à questionner et à appliquer ces concepts, car la modélisation et la simulation numérique sont des outils fondamentaux pour une carrière en sciences informatiques. En vous souhaitant un apprentissage riche et stimulant dans la suite de vos études !
