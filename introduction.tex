\chapter{Introduction}
    Ce syllabus est conçu pour guider l'étudiant·e dans l’exploration de la modélisation et de la simulation numérique, domaine actuellement incontournable pour aborder la complexité des systèmes dynamiques rencontrés dans de nombreuses disciplines scientifiques et techniques. Le cours de Modélisation et Simulation, dans bien des aspects, suit le cours de \textit{Calcul Formel et Numérique}, en y approfondissant les concepts et méthodes numériques appliqués aux systèmes dynamiques et aux équations différentielles, en mettant en pratique l'utilisation d’outils de programmation.

    Dans le cours de \textit{Calcul Formel et Numérique}, l'étudiant·e a été introduit·e aux bases de l’analyse numérique, au calcul formel et à la résolution numérique d’équations. Le cours de Modélisation et Simulation invite maintenant à appliquer et à étendre ces connaissances à des systèmes complexes, à interpréter leurs comportements et à visualiser leurs dynamiques dans des cas plus sophistiqués.

    Bien sûr, nous référons avant tout l'étudiant·e au cours de Mathématiques dispensé en première année de bachelier (\cite{mathf117}). La maîtrise extensive des concepts qui y sont présentés est fondamentale pour la bonne compréhension du contenu du cours de Modélisation et Simulation. La référence principale du cours est et restera le syllabus théorique (\cite{infof305}).

    L'objectif principal de ce document est d'offrir une méthode pour analyser des systèmes par la modélisation mathématique et par la simulation numérique. Chaque chapitre de ce syllabus a été conçu pour donner des bases théoriques solides, complétées de nombreux exercices pratiques, afin de renforcer la compréhension des concepts abordés par la mise en application. 
    Dans ce syllabus, nous abordons les points suivants:
    \begin{description}
        \item[Introduction aux Équations Différentielles:] dans ce premier chapitre, nous définissons les notions fondamentales permettant d'analyser et de résoudre certains types d'équations différentielles, plus spécifiquement, les équations différentielles ordinaires (EDO) linéaires et non-linéaires. Nous présentons les outils permettant de manipuler des équations du premier et du second ordre et de les résoudre analytiquement et numériquement, tout en construisant une intuition sur leur manière de modéliser et de simuler des systèmes dynamiques fondamentaux. Nous présentons aussi la librairie \codeword{SymPy} (\cite{Sympy2017}) permettant de résoudre des équations à l'aide de la programmation symbolique.
        \item[Dessin qualitatif de portraits de phases:] le comportement des systèmes décrits par ces équations différentielles ne nécessite pas forcément la résolution complète, et le comportement peut être évalué au moyen du \textit{portrait de phases}. Le portrait de phases est un outil visuel puissant qui vous permet de comprendre le comportement d'un système sans en déterminer l'ensemble des solutions analytiques. Ce chapitre présente une méthode pour analyser qualitativement le comportement des systèmes, identifier les points d’équilibre et dessiner des trajectoires dans l’espace des phases. Nous soulignons également le rôle essentiel des valeurs et vecteurs propres dans l'étude de ces comportements, notamment à l'aide du diagramme de Poincaré.
        \item[Simulation numérique en Python:] afin de tracer plus exactement le portrait de phases, des outils numériques peuvent être utilisés pour accélérer le processus de simulation. Dans ce chapitre, nous explorons la simulation numérique d’équations différentielles. Nous présentons des bibliothèques Python comme \codeword{NumPy} et \codeword{SciPy} (\cite{Numpy2020, Scipy2020}), pour simuler des systèmes dynamiques du premier et du second ordre. Les concepts de stabilité et d’intégration numérique sont abordés de manière appliquée, et ce chapitre présente des outils concrets pour aborder la complexité de systèmes difficilement résolubles analytiquement. 
        \item[Équations aux différences et systèmes à temps discrets:] ensuite, nous abordons  les systèmes à temps discret, au moyen d'équations aux différences. Ces modèles sont essentiels pour représenter des phénomènes échantillonnés ou itératifs.  Il existe des méthodes de résolution et d'analyse spécifiques à ce genre de systèmes.
        \item[Simulation Monte Carlo:] puis, nous montrons comment il est possible de simuler des systèmes dont les conditions initiales et/ou les paramètres sont soumis à l'aléatoire, en présentant la librairie \codeword{SimPy}. Nous étudions aussi comment en extraire des statistiques grâce à la notion de simulation Monte Carlo.
        \item[Simulations à événements discrets:] finalement, nous montrons comment il est possible de simuler des systèmes dont les événements peuvent survenir, en présentant la librairie \codeword{SimPy}. Nous étudions aussi comment en extraire des statistiques grâce à la notion de simulation Monte Carlo.
    \end{description}

    \paragraph{Pourquoi lire ce syllabus et réaliser les exercices ?}
    
    Ce syllabus n'est pas seulement un manuel d'apprentissage, mais une véritable boîte à outils pour tou·te·s les étudiant·e·s souhaitant maîtriser les techniques de modélisation et de simulation numérique. En réalisant les exercices et en expérimentant avec les exemples proposés, nous espérons le développement d'une intuition profonde du comportement des systèmes dynamiques et d'une maîtrise des méthodes mathématiques et informatiques existantes pour analyser des problèmes complexes. 

    L'auteur en a bien conscience, et il parle d'expérience: un syllabus comme le présent document ne présente aucun intérêt s'il est lu en survolant les équations, formules et autres symboles ésotériques utilisés. Plutôt, il est recommandé, pour chacune des lignes mathématiques apparaissant dans ce support, de prendre un bloc de feuilles, et de chercher, par l'exercice et la reproduction, à en comprendre tous les raisonnements. 

    Ce syllabus est donc plus qu’un document académique: c’est un guide pratique pour naviguer dans l’univers des systèmes dynamiques et vous préparer à des projets de recherche ou professionnels qui font appel à l'étude de systèmes dynamiques.

    \textit{Toute suggestion, erreur ou typo peut être signalée à l'adresse \href{mailto:pascal.tribel@ulb.be}{pascal.tribel@ulb.be} (et ceci est vivement encouragé!).}