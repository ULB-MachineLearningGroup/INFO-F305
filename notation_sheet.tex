\chapter{Notation Sheet}
On note
\begin{itemize}
    \item Un nombre par une lettre minuscule ($t$, $a$, \ldots).
    \item Un vecteur par une lettre minuscule chapeautée d'une flèche ($\overrightarrow{v}$, $\overrightarrow{y}$, \ldots)
    \item Une matrice par une lettre majuscule ($A$, $X$, \ldots).
    \item Une fonction par une lettre minuscule suivie de parenthèses ($f(t)$, $g(\overrightarrow{a})$, \ldots).
    \item La dérivée d'une fonction, en forme longue, par l'opérateur $\dd $ ($\frac{f(t)}{\dd t}$, $\frac{g(a)}{\dd a}$, \ldots), 
    \item Ou, en forme courte, par l'opérateur $\dot{}$ ($\dot{f}(t)$, $\dot{g}(\overrightarrow{a})$, \ldots).
    \item La $n^{\text{ème}}$ dérivée utilise la notation en exposant: $f^n(t)$.
\end{itemize}

Dans le cadre de résolution d'équations différentielles:
\begin{itemize}
    \item L'inconnue, une fonction, est indiquée par une minuscule, et le paramètre est généralement $t$: $x(t)$.
    \item Une solution particulière numérotée $k$ est notée par $x^{(k)}(t)$. Notez la différence avec la notation de la $k^{\text{ème}}$ dérivée, qui n'utilise pas de parenthèses autour de l'exposant.
\end{itemize}